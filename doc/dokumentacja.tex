\documentclass[10pt,a4paper]{article}
\usepackage[polish]{babel}
\usepackage[utf8]{inputenc}
\usepackage{polski}
\usepackage{indentfirst}
\usepackage{mathtools} 
\usepackage[a4paper,margin=2cm]{geometry}
\usepackage{tabularx}
\usepackage{graphicx}
\usepackage{float}
\frenchspacing

\usepackage{titlesec}
\titlelabel{\thetitle.\quad}

\begin{document}
	\title{Kryptografia - projekt  \\  Algorytm RSA}
	\author{Piotr Janus, Kamil Piszczek}
	\date{}
	\maketitle

\section{Opis metody}
Algorytm RSA jest jednym z pierwszych praktycznych kryptosystemów korzystających z koncepcji klucza publicznego i jest on powszechnie stosowany do zabezpieczenia transmisji danych. W tym kryptosystemie, klucz szyfrujący jest publiczny i różni się od klucza deszyfrującego, który jest tajny. Ta asymetria kluczy wynika z faktu, że niezwykle trudno jest rozłożyć na czynniki iloczyn dwóch bardzo dużych liczb pierwszych. 

Nazwa algorytmu RSA pochodzi od pierwszych liter nazwisk jego twórców - Ron Rivesta, Adi Shamira oraz Leonarda Adlemana, którzy po raz pierwszy opisali go w 1977r.

Algorytm RSA jest stosunkowo wolnym algorytm i z tego powodu jest on rzadziej używany do bezpośrednio szyfrowania danych użytkownika. W większości przypadków, przy pomocy RSA przekazywane są klucze do symetrycznych algorytmów szyfrujących, które potem mogą być wykorzystywane do stosunkowo bezpiecznej wymiany danych z dużo wyższą prędkością.

\subsection{Generacja kluczy} \label{key_gen}

Zarówno klucz prywatny jak i publiczny, które wykorzystywane są w algorytmie RSA tworzone są według poniższej procedury:

\begin{enumerate}
\item Wybierz dwie różne liczby pierwszy $p$ oraz $q$.
\item Oblicz $n = pq$. $n$ jest modułem (ang. modulus) zarówno klucza prywatnego jak i publicznego. Oznacza również ich długość (zazwyczaj podawana w bitach).
\item Oblicz funkcję Eulera - tzw. tocjent: $\phi(n) = \phi(p) \phi(q) = (p-1)(q-1)$
\item Wybierz liczbę $e$ taką że: $1<e<\phi(n)$ oraz $NWD(e,phi(n))=1$.
\item Ustal liczbę $d$ taką że:
\begin{equation} \label{ed_cond_eq}
d e \equiv 1 \mod \phi(n).
\end{equation}
\end{enumerate}

Klucz publiczny definiowany jest jako para liczb $(n, e)$, gdzie $n$ to moduł klucza, natomiast $e$ to jego wykładnik.

Klucz prywatny definiowany jest analogicznie jako para liczb $(n, d)$, gdzie $n$ to moduł klucza, natomiast $d$ to jego wykładnik.

\subsection{Szyfrowanie i deszyfrowanie}
Załóżmy, że chcemy zaszyfrować wiadomość $M$. Na początku dzielimy ją na mniejsze $m$ bloki według łatwego do odwrócenia schematu. Następnie obliczamy wartość zaszyfrowanej wiadomości $c$ według wzoru:
\begin{equation} \label{encode_eq}
c \equiv m^e \mod n
\end{equation} 

Załóżmy, że chcemy odszyfrować szyfrogram $c$ i odzyskać wiadomość $m$. Obliczamy wartość odszyfrowanego fragmentu wiadomości według wzoru:
\begin{equation} \label{decode_eq}
m \equiv c^d \mod n
\end{equation}

\subsection{Prosty przykład}

Postępujemy według przepisu \ref{key_gen}:

\begin{enumerate}
\item $p = 61$ oraz $q = 53$.
\item $n = 61 \times 53 = 3233 $.
\item $\phi(3233) = (61-1)(53-1) = 3120$.
\item $e=17$, ponieważ $1<17<3120$ oraz $NWD(3120, 17) = 1$.
\item $d=2753$, ponieważ $2753 \times 17 \equiv 1 \mod 3120$.
\end{enumerate}

Publiczny klucz to para: $(n, e)=(3233,17)$.
Prywatny klucz to para: $(n, e)=(3233,2753)$.

Szyfrogram dla wiadomości $m = 65$ obliczamy z podanego wzoru (\ref{encode_eq}):
\begin{equation} 
m^e = 65^{17} \equiv 2790 \mod 3233
\end{equation}

Oryginalną wiadomość możemy odszyfrować za pomocą wzoru (\ref{decode_eq}):
\begin{equation} 
c^d = 2790^{2753} \equiv 65 \mod 3233
\end{equation}

\subsection{Dowód poprawności}
Dowód poprawności algorytmu RSA opiera się na małym twierdzeniu Fermata. Jeżeli $p$ jest liczbą pierwszą i $a$ nie jest podzielne przez $p$ to wtedy:

\begin{equation} \label{fermat_eq}
a^{p-1} \equiv 1 \mod p
\end{equation}

Chcemy pokazać, że $m^{ed} \equiv m \mod pq$ dla każdej liczby całkowitej $m$. Zgodnie z wcześniejszymi założeniami $p$ oraz $q$ są różnymi liczbami pierwszymi, natomiast $e$ oraz $d$ to liczby całkowite spełniające warunek (\ref{ed_cond_eq}).
Ponieważ: $\phi(pq) = (p-1)(q-1)$, to na mocy (\ref{ed_cond_eq}) możemy stwierdzić, że dla każdego dodatniego i całkowitego $h$ zachodzi równanie:

\begin{equation}  \label{h_cond_eq}
ed-1=h(p-1)(q-1)
\end{equation}

Do zweryfikowania, czy liczby takie jak $m$ czy $m^{ed}$ przystają do siebie modulo $pq$ wystarczy sprawdzić, czy przystają one modulo $p$ oraz modulo $q$ oddzielnie. Wynika to, z chińskiego twierdzenia o reszcie.

Do pokazania, że $m^{ed} \equiv m \mod p$ rozważymy dwa przypadki. 

\begin{enumerate}
\item $m \equiv 0 \mod p$

Wtedy:
\begin{equation} 
m^{ed} \equiv 0 \equiv m \mod p
\end{equation}

\item $m \not\equiv 0 \mod p$

Przekształcamy przy pomocy zależności (\ref{h_cond_eq}) oraz twierdzenia Fermata (\ref{fermat_eq}):

\begin{equation} 
m^{ed} = m^{ed-1}m = m^{h(p-1)(q-1)}m = (m^{p-1})^{h(q-1)}m \equiv 1^{h(q-1)}m \equiv m \mod p
\end{equation}

\end{enumerate}

Analogiczne rozumowanie można przeprowadzić dla czynnika $q$. Te dwie zależności dowodzą poprawności algorytmu RSA:

\begin{equation}
m^{ed} \equiv m \mod pq
\end{equation}

\subsection{Próby złamania}

\section{Metody użyte wewnątrz algorytmu}
\subsection{Generacja bardzo dużych liczb pierwszych}
\subsection{Odwrotność modulo}
\subsection{Największy wspólny dzielnik dla dużych liczb}
\subsection{Operacja szyfrowania}
\subsection{Operacja deszyfrowania}

\section{Implementacja}
\subsection{Użyty język i biblioteki}
\subsection{Funkcjonalność}
\subsection{Interfejs użytkownika}


	
\end{document}